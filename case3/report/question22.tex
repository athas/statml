\section*{Question 2.2}

The code for this question is in the file \texttt{22.R}.

\subsection*{Question 2.2.1}

\includegraphics[width=8cm]{../img/question22-plot.pdf}

%\texttt{Deliverables:} 
%\begin{itemize}
%	\item \texttt{daniel: } plot of the data with visualization of the SVM model 
%\end{itemize}

\subsection*{Question 2.2.2}

\subsubsection*{$C=0.001$}
\includegraphics[width=8cm]{../img/question-2-2-1-plot1.pdf}
\subsubsection*{$C=10$}
\includegraphics[width=8cm]{../img/question-2-2-1-plot2.pdf}

As a free support vector is a vector where $0<alpha<C$, clearly
decreasing the value of $C$ will result in more bound support vectors,
while increasing $C$ will lead to more free support vectors.

%\texttt{Deliverables:} 
%\begin{itemize}
%	\item \texttt{daniel,med søtte fra troels} short discussion of how the SVM model changes depending on the choice of C 
%\end{itemize}

\subsection*{Question 2.2.3}

\begin{tabular}{|l|r|}
\textbf{Data set} & \textbf{Number of support vectors} \\\hline
\texttt{knollC-train100.dt} & 66 \\
\texttt{knollC-train200.dt} & 119 \\
\texttt{knollC-train400.dt} & 59 \\
\end{tabular}

%\texttt{Deliverables:} 
%\begin{itemize}
	
%	\item \texttt{daniel: } table showing the numbers of support vectors for the three datasets
	
%	\item \texttt{daniel,med søtte fra troels} brief theoretical discussion of the scaling behavior of SVMs w.r.t. the number of training patterns 
%\end{itemize}
